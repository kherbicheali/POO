\begin{appendices}
\chapter*{\textsc{Annexe A : MyLoopHandlerClass.h}}
	\addcontentsline{toc}{chapter}{\textsc{Annexe A}}		
	
	\begin{lstlisting}	
	 
#pragma once
#include "mmtimer.h"
#include <math.h>
#include <iostream>
#include <fstream>

class MyLoopHandlerClass :
	public CMultimediaTimerCallback
{
private: 
	int T1,Tf,T0;
	float Vmax;
	std::ofstream out;
public:
	MyLoopHandlerClass(int,int,int,float);
public:
	virtual ~MyLoopHandlerClass(void);
public:
	bool OnLoop();
	void OnStart();
	void Stopped ();
	void OnStopped ();
};
//int MyLoopHandlerClass *T0

	\end{lstlisting}
	
	\chapter*{\textsc{Annexe B : MyLoopHandlerClass.cpp}}
	\addcontentsline{toc}{chapter}{\textsc{Annexe B}}		

	\begin{lstlisting}	
	 
#include "stdafx.h"
#include "MyLoopHandlerClass.h"
#include "MMTimer.h"
//#include <chrono>
//#include <boost/thread/thread.hpp>
using namespace std;

MyLoopHandlerClass::MyLoopHandlerClass(int t0,int t1,int tf, float vmax) : out("trajectoire.txt")
{
	this -> T1=t1; //ms
	this -> Tf=tf; //ms
	this -> Vmax=vmax; // Vmax = 0.1 cm/ms
	this -> T0=t0;
	
}

MyLoopHandlerClass::~MyLoopHandlerClass(void)
{
}

 bool MyLoopHandlerClass:: OnLoop (void)
 {	
	int t = GetTickCount()-T0-20;
	double D,V; // distance et Vitesse 
	
	if (t <= T1)
	{
		D =  0.5*Vmax*t*t/T1; // cm
		V= t*Vmax/T1;  
	}
	if((t>T1) && (t<=Tf-T1))
	{
		D=  0.5*Vmax*T1+Vmax*(t-T1); // cm
		V=Vmax;
	} 
	if(t>Tf-T1 && t<=Tf)
	{
		D = 0.5*Vmax*T1+Vmax*(Tf-T1-T1)+ 0.5*Vmax/T1*((Tf-T1)*(Tf-T1)-t*t)+Vmax*Tf/T1*(t-(Tf-T1)); //cm 
		V=-Vmax+((Vmax/(Tf-T1-Tf))*(t-Tf-T1));
		
	}
	cout << t << " ms\t" << V << " cm/s\t"<< D <<"cm"<< std::endl;
	out << t << "\t" << D << "\t" <<V << std::endl;

	return true;
}

 void MyLoopHandlerClass:: OnStopped ()
 {
	 out.close();
 }
	 
	\end{lstlisting}
	
	\chapter*{\textsc{Annexe B : maeva\text{\_}sql.cpp}}
	\addcontentsline{toc}{chapter}{\textsc{Annexe B}}		

	\begin{lstlisting}	
	 
	// maeva_sql.cpp : Defines the entry point for the console application.
//

#include "stdafx.h"
#include <iostream>
#include "MMTimer.h"
#include "libRobotPekee.h"
#include "MyLoopHandlerClass.h"

using namespace std;

//Declarations globales au programme
//

WRobotPekee Pekee;

// -- Commenter en fonction pour selectionner le mode de commande
// Configuration mode Simulation
BUFFER experience="simulation sur le RSL.";
BUFFER szRSLHostName="localhost";

// Configuration mode Reel
//BUFFER experience="commande du robot reel.";
//BUFFER szRSLHostName="130.120.12.101";
class  handler:public CMultimediaTimerCallback{
public:
	int GetTickCount (void);
	void Sleep(unsigned int t );
};


int main(int argc, char* argv[])
{
	
int TT1=0;
int TTf=0;
float VVmax;
	printf("Coucou\n");
	
	cout << "Vmax = " ;
	cin >> VVmax ;
	cout << "t1 = " ;
	cin >> TT1;
	cout << "tf = ";
	cin >> TTf;
	// Connexion au robot
	//
	std::cout << "Essai de connection a " << szRSLHostName << std::endl;
	if(Pekee.Connect(szRSLHostName))
    {
		std::cout << "La connexion est etablie" << std::endl;


	  // --------------------------------- Initialisation du Timer
		int TT0=GetTickCount();
		MyLoopHandlerClass *handler = new MyLoopHandlerClass (TT0,TT1,TTf,VVmax);
		CMultimediaTimer timer(*handler);

		timer.Start(20,1);

	  // --------------------------------- Tracage de la Position

		while (!kbhit()) {Sleep(1); Pekee.SetSpeed(100,0) ;}
 
	  // --------------------------------- Arret du Timer
		
		timer.Stop();

	  // Arret de la connexion au Robot

		Pekee.Disconnect();
    }
	else
    {
		std::cout << "La connexion n'a pu etre etablie." << std::endl;
    }
	return 0;
}
	 
	\end{lstlisting}
	
\end{appendices}
